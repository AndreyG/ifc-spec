\label{sec:ifc-source-location}

At various places, especially in the description of declarations (\secref{sec:ifc-decls}), it is necessary to specify 
locations in input source code.  Ideally, there should be distinction between source location as seen by the
user (before macro expansions), a span of source location, and location traking macro expansions.
For the time being, source location is described in a fairly simplistic way using the unsophisticated
structure \type{SourceLocation} defined as follows:
%
\begin{figure}[h]
	\centering
	\structure{
		\DeclareMember{line}{LineIndex} \\
		\DeclareMember{column}{Column} \\
	}
	\caption{Structure of a source location}
	\label{fig:ifc-source-location-structure}
\end{figure}


\noindent
The \field{line} field is of type \type{LineIndex} defined as \newtype{LineIndex}{32}
A value of this type is an index into the partition of file and line (\secref{sec:ifc-file-and-line-location}).

\noindent
The \field{column} field is of type \type{Column} defined as \newtype{Column}{32}
A value of this type indicates the position of the character from the beginning of the \field{line}.
Columns are numbered from 0.

\section{File and line}
\label{sec:ifc-file-and-line-location}

The file and line source location information is collectively stored in a dedicated partition.
Each entry in that partition is a structure with the following components:
%
\begin{figure}[h]
	\centering
	\structure{
		\DeclareMember{file}{NameIndex} \\
		\DeclareMember{line}{LineNumber} \\
	}
	\caption{Structure of a file-and-line location information}
	\label{fig:ifc-file-and-line-structure}
\end{figure}
%
The \field{file} field designates the name of the input source file (sec:ifc-source-file-name).

\noindent
The \field{line} field is the line number in the designated source file. Lines are numbered from 1.
A line number is of type \type{LineNumber} defined as \newtype{LineNumber}{32}

\partition{src.line}
