\label{sec:ifc-heaps}

At various places, there is a need to describe a sequence of objects of a given (common) sort but of differing kinds.  
For example, a namespace contains only declarations, but those declarations can be of different kinds; e.g. function declaration, variable declaration,
template declaration, etc.  Those tables are represented as sequences of homogenous indices of one sort: a slice of a heap of indices.

\section{Heap structures}
\label{sec:ifc:heap-structures}

\subsection{Declaration heap}
\label{sec:ifc-decl-heap}

The declarations heap is a partition consisting entirely of \type{DeclIndex} values.

\partition{heap.decl}



\subsection{Type heap}
\label{sec:ifc-type-heap}

The types heap is a partition consisting entirely of \type{TypeIndex} values.

\partition{heap.type}


\subsection{Statement heap}
\label{sec:ifc-stmt-heap}

The statement heap is a partition consisting entirely of \type{StmtIndex} values.

\partition{heap.stmt}


\subsection{Expression heap}
\label{sec:ifc-expr-heap}

The expression heap is a partition consisting entirely of \type{ExprIndex} values.

\partition{heap.expr}


\subsection{Syntax heap}
\label{sec:ifc-syntax-heap}

The syntax heap is a partition consisting entirely of \type{SyntaxIndex} values.

\partition{heap.syn}

\subsection{Form heap}
\label{sec:ifc-form-heap}

The preprocessing form heap is a partition consisting entirely of \type{FormIndex} values.

\partition{heap.form}



\subsection{Chart heap}
\label{sec:ifc-chart-heap}

The chart heap is a partition consisting entirely of \type{ChartIndex} values.

\partition{heap.chart}

\subsection{Attribute heap}
\label{sec:ifc-attr-heap}


The attribute heap is a partition consisting entirely of \type{AttrIndex} values.

\partition{heap.attr}



