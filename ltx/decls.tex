\label{sec:ifc-decls}

Declarations are indicated by abstract declaration references.  This document
uses \type{DeclIndex} to designate a typed abstract reference to a
declaration.  Like all abstract references, it is a 32-bit value
\begin{figure}[htbp]
  \centering
  \absref{5}{DeclSort}
  \caption{\type{DeclIndex}: Abstract reference of declaration}
  \label{fig:ifc-decl-index}
\end{figure}

\begin{SortEnum}{DeclSort}
	\enumerator{VendorExtension}
	\enumerator{Enumerator}
	\enumerator{Variable}
	\enumerator{Parameter}
	\enumerator{Field}
	\enumerator{Bitfield}
	\enumerator{Scope}
	\enumerator{Enumeration}
	\enumerator{Alias}
	\enumerator{Temploid}
	\enumerator{Template}
	\enumerator{PartialSpecialization}
	\enumerator{ExplicitSpecialization}
	\enumerator{ExplicitInstantiation}
	\enumerator{Concept}
	\enumerator{Function}
	\enumerator{Method}
	\enumerator{Constructor}
	\enumerator{InheritedConstructor}
	\enumerator{Destructor}
	\enumerator{Reference}
	\enumerator{UsingDeclaration}
	\enumerator{UsingDirective}
	\enumerator{Friend}
	\enumerator{Expansion}
	\enumerator{DeductionGuide}
	\enumerator{Barren}
	\enumerator{Tuple}
	\enumerator{SyntaxTree}
	\enumerator{Intrinsic}
	\enumerator{Property}
	\enumerator{OutputSegment}
\end{SortEnum}

\paragraph{Note:}
The individual values a \type{DeclSort} enumerator is subject to change
at any moment until the design stabilizes.

\section{Declaration vocabulary types}
\label{sec:ifc-decl-support-types}

The description of declarations uses a set of common types values as described below.

\subsection{Access specifiers}
\label{sec:ifc-access-specifier}

Every non-local declaration has an access specifier, of type:
\begin{typedef}{Access}{}
	enum class Access : uint8_t {
		None,				// No access specifier
		Private,				// "private" for scope member
		Protected,			// "protected" for scope member
		Public,				// "public" for scope member
	};
\end{typedef}

\subsection{Basic specifiers}
\label{sec:ifc-basic-specifiers}

Certain cumulative properties common to all declarations are described by the bitmask type \type{BasicSpecifiers}:
%
\begin{typedef}{BasicSpecifiers}{}
	enum class BasicSpecifiers : uint8_t {
		Cxx                     = 0,        // C++ language linkage
		C                       = 1 << 0,   // C language linkage
		Internal                = 1 << 1,   // 
		Vague                   = 1 << 2,   // Vague linkage, e.g. COMDAT, still external
		External                = 1 << 3,   // External linkage.
		Deprecated              = 1 << 4,   // [[deprecated("foo")]]
		InitializedInClass      = 1 << 5,   // defined or initialized in a class
		NonExported             = 1 << 6,   // Not explicitly exported
		IsMemberOfGlobalModule  = 1 << 7    // member of the global module
	};
\end{typedef}
%

\note{The definition of \type{BasicSpecifiers} may change in the future, and may in fact be part of \type{Declaration} (\secref{sec:ifc-scope-member}).
The numerical values assigned to these symbolic constants are subject to change.
}

\subsection{Reachable semantic properties}
\label{sec:ifc-reachable-properties}

In certain circumstances, the IFC stores more information than the bare minimum
required by the ISO C++ Modules specification.  In such cases, it is necessarilyto know 
which semantic properties are reachable, outside the owning module, to the importers.
In other circumstances, known such additional information is useful in performing
additional checks such as ODR violation detection.  The availability of such
supplementary information is indicated by the bitmask \type{ReachableProperties}
\begin{typedef}{ReachableProperties}{}
    enum class ReachableProperties : uint8_t {
        None                = 0,        // nothing beyond name, type, scope.
        Initializer         = 1 << 0,   // IPR-initializer exported.
        DefaultArguments    = 1 << 1,   // function or template default arguments exported
        Attributes          = 1 << 2,   // standard attributes exported.
        All                 = 0xff,     // Everything.
    };
\end{typedef}
%



\subsection{Object traits}
\label{sec:ifc-object-traits}

Certain cumulative properties common to all data/object declarations are described by the bitmask type \type{ObjectTraits}
%
\begin{typedef}{ObjectTraits}{}
	enum class ObjectTraits : uint8_t {
		None					= 0,
		Constexpr				= 1 << 0,
		Mutable				= 1 << 1,
		ThreadLocal			= 1 << 2,
		Inline					= 1 << 3,
		InitializerExported	= 1 << 4,
		Vendor					= 1 << 7,
	};
\end{typedef}
%

\note{The definition of \type{ObjectTraits} is subject to change.
The numerical values assigned to these symbolic constants are subject to change.}

\subsection{Vendor traits}
\label{sec:ifc-msvc-trait-bitset}

Declarations of certain entities may be endowed with vendor-specific traits.
The MSVC-specific traits are defined by the following enumeration
\begin{typedef}{MsvcTraits}{}
	enum class MsvcTraits : uint32_t {
		None					= 0,
		ForceInline			= 1 << 0,
		Naked					= 1 << 1,
		NoAlias				= 1 << 2,
		NoInline				= 1 << 3,
		Restrict				= 1 << 4,
		SafeBuffers			= 1 << 5,
		DllExport				= 1 << 6,
		DllImport				= 1 << 7,
		CodeSegment			= 1 << 8,
		Novtable				= 1 << 9,
		IntrinsicType			= 1 << 10,
		EmptyBases			= 1 << 11,
		Process				= 1 << 12,
		Allocate				= 1 << 13,
		SelectAny				= 1 << 14,
		Comdat				= 1 << 15,
		Uuid					= 1 << 16,
	};
\end{typedef}

\subsection{Parameter Level}

A template declaration can have many nesting levels.  This is the case of member templates of class templates; that is a member of a class template, that is itself a template.
Parameter nesting level starts from 1.  The nesting level is given by a value of type \type{ParameterLevel} defined as
\newtype{ParameterLevel}{32}

\subsection{Parameter Position}

A parameter at a given level can be identified by its position in its enclosing parameter list.  The  position of a template parameter is given by a value of type \type{ParameterPosition}, defined as \newtype{ParameterPosition}{32}

\section{Declaration structures}
\label{sec:ifc:decl-structures}

\subsection{\valueTag{DeclSort::VendorExtension}}
\label{sec:ifc:DeclSort:VendorExtension}
A \type{DeclIndex} value with tag \valueTag{DeclSort::VendorExtension} represents an abstract reference to a vendor-specific declaration.
This tag value is reserved for encoding vendor-specific extensions.

\partition{decl.vendor-extension}

\subsection{\valueTag{DeclSort::Enumerator}}
\label{sec:ifc:DeclSort:Enumerator}

A \type{DeclIndex} value with tag \valueTag{DeclSort::Enumerator} represents
an abstract reference to an enumerator declaration.
The \field{index} field is an index into the enumerator declaration partition.
Each entry in that partition is a structure with the following components:
a \field{name} field, a \field{locus} field, a \field{type} field, an \field{initializer} field,
a \field{specifier} field, and an \field{access} field.
%
\begin{figure}[H]
	\centering
	\structure{
		\DeclareMember{name}{TextOffset} \\
		\DeclareMember{locus}{SourceLocation} \\
		\DeclareMember{type}{TypeIndex} \\
		\DeclareMember{initializer}{ExprIndex} \\
		\DeclareMember{specifier}{BasicSpecifiers} \\
		\DeclareMember{access}{Access} \\
	}
	\caption{Structure of an enumerator declaration}
	\label{fig:ifc-enumerator-decl-structure}
\end{figure}
%
The \field{name} field denotes the C++ source-level name of the enumerator.
The \field{locus} field denotes the source location.
The \field{type} field denotes the type of the enumerator.
The \field{initializer} field denotes the value or the initializer of the enumerator.
The \field{specifier} field denotes the specifiers of the enumerator.
The \field{access} field denotes the C++ source-level access specifier of the enumerator.

\partition{decl.enumerator}

\subsection{\valueTag{DeclSort::Variable}} 
\label{sec:ifc:DeclSort:Variable}

A \type{DeclIndex} value with tag \valueTag{DeclSort::Variable} represents an abstract
reference to a variable declaration.  Note that static data members are also
semantically variables and are represented as such.
The \field{index} field is an index into the variable declaration partition.
Each entry in that partition is a structure with the following components:
a \field{name} field, a \field{locus} field, a \type{field}, a \field{home\_scope} field,
an \field{initializer} field, an \field{alignment} field, a \field{specifier} field, a \field{traits}
field, and an \field{access} field.
%
\begin{figure}[H]
	\centering
	\structure[text width = 14em]{
		\DeclareMember{name}{NameIndex} \\
		\DeclareMember{locus}{SourceLocation} \\
		\DeclareMember{type}{TypeIndex} \\
		\DeclareMember{home\_scope}{DeclIndex} \\
		\DeclareMember{initializer}{ExprIndex} \\
		\DeclareMember{alignment}{ExprIndex} \\
		\DeclareMember{traits}{ObjectTraits} \\
		\DeclareMember{specifier}{BasicSpecifiers} \\
		\DeclareMember{access}{Access} \\
		\DeclareMember{properties}{ReachableProperties} \\
	}
	\caption{Structure of a variable declaration}
	\label{fig:ifc-variable-decl-structure}
\end{figure}
%
The \field{name} field denotes the name of the variable.  Note that it can be a plain identifier (a \type{TextOffset} into the string table), or something
as elaborated as a template-id (for specializations of variable templates).
The \field{locus} field denotes the source location.
The \field{type} field denotes the C++ source-level type of the variable.
The \field{home\_scope} field denotes the scope declaration that holds the object the variable designates.  The \field{home\_scope} 
is not necessarily the lexical scope of a variable: for instance, a block-scope 'extern' declaration of a variable names a variable whose 
home scope in the nearest enclosing namespace scope.
The \field{initializer} field denotes the initializer expression in the variable declaration.
The \field{alignment} field denotes the alignment of the variable.
The \field{specifier} field denotes the declarations specifiers of the variable.
The \field{traits} field denotes additional traits associated with the variable.
The \field{properties} field indicates which semantic properties are reachable 
to the importers.

\partition{decl.variable}


\subsection{\valueTag{DeclSort::Parameter}}
\label{sec:ifc:DeclSort:Parameter}

A \type{DeclIndex} value with \field{tag} \valueTag{DeclSort::Parameter} is an abstract reference to 
either a function parameter or a template parameter declaration.
The \field{index} field is an index into the parameter declaration partition.  
Each entry in that partition is a structure with the following layout
%
\begin{figure}[H]
	\centering
	\structure[text width = 14em]{
			\DeclareMember{name}{TextOffset} \\
			\DeclareMember{locus}{SourceLocation} \\
			\DeclareMember{type}{TypeIndex} \\
			\DeclareMember{constraint}{ExprIndex} \\
			\DeclareMember{initializer}{ExprIndex} \\
			\DeclareMember{level}{ParameterLevel} \\
			\DeclareMember{position}{ParameterPosition} \\
			\DeclareMember{sort}{ParameterSort} \\
			\DeclareMember{properties}{ReachableProperties} \\
			\DeclareMember{pack}{bool} \\
		}
	\caption{Structure of a template parameter declaration}
	\label{fig:ifc-template-parameter-structure}
\end{figure}
%
and these meanings of the fields:
\begin{itemize}
	\item \field{name} denotes the name of the template parameter.  If null, the template parameter was unnamed in the source input.
	\item \field{locus} denotes the location of the template parameter.
	\item \field{type} designates the type of the parameter.
	\item \field{constaint} designates the concept predicate used to declare this parameter, if the abbreviated form was used at the input source level.
	\item \field{initializer} designates the corresponding default argument, if any.
	\item \field{level} denotes the nesting level of this parameter
	\item \field{position} denotes the position of this parameter in the parameter list
	\item \field{sort} denotes the sort of parameter (function-parameter vs template-parameter)
	\item \field{properties} denotes the set of reachable properties of this parameter.
	\item \field{pack} is true if this parameter was declared as a parameter pack.  This field is no longer needed.
\end{itemize}

\note{This representation will change in the future as it is currently too irregular and too tightly coupled with VC++ internal representation oddities.}

\partition{decl.parameter}

\subsubsection{Parameter sort}
\label{sec:ifc-parameter-sort}

The various notions of parameters (function parameter, type template parameter, non-type template parameter, template template parameter) are described by:
\begin{typedef}{ParameterSort}{}
	enum class ParameterSort : uint8_t {
		Object,					// Function parameter
		Type,						// Type template parameter
		NonType,					// Non-type template parameter
		Template,					// Template template parameter
	};
\end{typedef}

\subsection{\valueTag{DeclSort::Field}} 
\label{sec:ifc:DeclSort:Field}

A \type{DeclIndex} value with tag \valueTag{DeclSort::Field} represents 
an abstract reference to the representation of a non-static data member declaration.
The \field{index} field is an index into the field declaration partition.
Each entry in that partition is a structure with the following components:
a \field{name} field, a \field{locus} field, a \field{type} field, a \field{home\_scope} field,
an \field{initializer} field, an \field{alignment} field, a \field{specifier} field,
a \field{traits} field, and an \field{access} field.
%
\begin{figure}[H]
	\centering
	\structure[text width = 14em]{
		\DeclareMember{name}{TextOffset} \\
		\DeclareMember{locus}{SourceLocation} \\
		\DeclareMember{type}{TypeIndex} \\
		\DeclareMember{home\_scope}{DeclIndex} \\
		\DeclareMember{initializer}{ExprIndex} \\
		\DeclareMember{alignment}{ExprIndex} \\
		\DeclareMember{traits}{ObjectTraits} \\
		\DeclareMember{specifier}{BasicSpecifiers} \\
		\DeclareMember{access}{Access} \\
		\DeclareMember{properties}{ReachableProperties} \\
	}
	\caption{Structure of a field declaration}
	\label{fig:ifc-field-decl-structure}
\end{figure}
%
The \field{name} field denotes the name of the non-static data member.
The \field{locus} field denotes the source location.
The \field{type} field denotes the C++ source-level type of the non-static data member.
The \field{home\_scope} field denotes the scope declaration that holds the member declaration.  
The \field{initializer} field denotes the initializer expression in the member declaration.
The \field{alignment} field denotes the alignment of the non-static data member.
The \field{specifier} field denotes the declarations specifiers of the non-static data member.
The \field{traits} fields denotes additional traits associated with the non-static data member.


\partition{decl.field}

\subsection{\valueTag{DeclSort::Bitfield}}
\label{sec:ifc:DeclSort:Bitfield}

A \type{DeclIndex} value with tag \valueTag{DeclSort::Bitfield} represents 
an abstract reference to the representation of a bitfield declaration.
The \field{index} field is an index into the bitfield declaration partition.
Each entry in that partition is a structure with the following components:
a \field{name} field, a \field{locus} field, a \field{type} field, a \field{home\_scope} field,
a \field{width} field,
an \field{initializer} field, a \field{specifier} field,
a \field{traits} field, and an \field{access} field.
%
\begin{figure}[H]
	\centering
	\structure[text width = 14em]{
		\DeclareMember{name}{TextOffset} \\
		\DeclareMember{locus}{SourceLocation} \\
		\DeclareMember{type}{TypeIndex} \\
		\DeclareMember{home\_scope}{DeclIndex} \\
		\DeclareMember{width}{ExprIndex} \\
		\DeclareMember{initializer}{ExprIndex} \\
		\DeclareMember{traits}{ObjectTraits} \\
		\DeclareMember{specifier}{BasicSpecifiers} \\
		\DeclareMember{access}{Access} \\
		\DeclareMember{properties}{ReachableProperties} \\
	}
	\caption{Structure of a bitfield declaration}
	\label{fig:ifc-bitfield-decl-structure}
\end{figure}
%
The \field{name} field denotes the name of the bitfield.
The \field{locus} field denotes the source location.
The \field{type} field denotes the C++ source-level type of the bitfield.
The \field{home\_scope} field denotes the scope declaration that holds the bitfield declaration.  
The \field{width} field denotes the number bits specified in the bitfield declaration.
The \field{initializer} field denotes the initializer expression in the bitfield declaration.
The \field{specifier} field denotes the declarations specifiers of the bitfield.
The \field{traits} fields denotes additional traits associated with the bitfield.

\partition{decl.bitfield}

\subsection{\valueTag{DeclSort::Scope}}
\label{sec:ifc:DeclSort:Scope}

A \type{DeclIndex} abstract reference with tag \valueTag{DeclSort::Scope} designates a class-type or a namespace definition.
The \field{index} field of that abstract reference is an index into the scope declaration partition.
Each entry in that partition is a structure with the following components:
%
\begin{figure}[H]
	\centering
	\structure[text width = 14em]{
		\DeclareMember{name}{NameIndex} \\
		\DeclareMember{locus}{SourceLocation} \\
		\DeclareMember{type}{TypeIndex} \\
		\DeclareMember{base}{TypeIndex} \\
		\DeclareMember{initializer}{ScopeIndex} \\
		\DeclareMember{home\_scope}{DeclIndex} \\
		\DeclareMember{alignment}{ExprIndex} \\
		\DeclareMember{pack\_size}{PackSize} \\
		\DeclareMember{specifiers}{BasicSpecifiers} \\
		\DeclareMember{traits}{ScopeTraits} \\
		\DeclareMember{access}{Access} \\
		\DeclareMember{properties}{ReachableProperties} \\
	}
	\caption{Structure of a scope declaration}
	\label{fig:ifc-scope-decl-structure}
\end{figure}


\noindent
The \field{name} field designates the name of the scope.
The \field{locus} field designates the source location.
The \field{type} field indicates the kind (\secref{sec:ifc:TypeSort:Fundamental}) of scope:
\begin{itemize}
  \item \code{TypeBasis::Struct} means the scope was declared as "\code{struct}"
  \item \code{TypeBasis::Class} means the scope was declared as "\code{class}"
  \item \code{TypeBasis::Union} means the scope was declared as "\code{union}"
  \item \code{TypeBasis::Namespace} means the scope was declared as "\code{namespace}"
  \item \code{TypeBasis::Interface} means the scope was declared as "\code{\_\_interface}"
\end{itemize}
Any other value is invalid.

\noindent
The \field{base} field designates the base class(es) in case of inheritance.
The \field{initializer} field designates the body of the scope definition, e.g. the sequence of declarations.  
Note that valid \type{ScopeIndex} values start from 1, and 0 indicates absence of scope, e.g. an incomplete class type.
The \field{home\_scope} field designates the declaration of the enclosing scope.
The \field{alignment} field designates the alignment value of the scope, in case of class-type.
The \field{pack\_size} field designates the packing value applied to the layout of the scope, in case of class-type.
The \field{specifiers} field indicates the (cumulative) basic declaration specifiers that hold for the scope.
The \field{traits} field designates scope-specific properties of the scope.
The \field{access} field designates the access specifier of the scope declaration.
The \field{properties} field designates the set of reachable semantic properties.

\partition{decl.scope}

\subsubsection{Scope traits}
\label{sec:ifc-scope-traits}

Properties specific to scope entities are described by values of the bitmask type \type{ScopeTraits}:
%
\begin{typedef}{ScopeTraits}{}
	enum class ScopeTraits : uint8_t {
		None			= 0,
		Unnamed		= 1 << 0,
		Inline			= 1 << 1,
		InitializerExported	= 1 << 2,
		ClosureType	= 1 << 3,
		Final = 1 << 4,
		Vendor			= 1 << 7,
	};
\end{typedef}
%
with the following meaning:
\begin{itemize}
  \item \code{ScopeTraits::None}: No scope traits.
  \item \code{ScopeTraits::Unnamed}: the scope is unnamed in the input source code.
  \item \code{ScopeTraits::Inline}: valid only for namespaces.  The namespace is declared \code{inline}.
  \item \code{ScopeTraits::InitializedExported}: valid only if the definition of this scope entity is lexically exported, in particular this indicates whether completeness of types is exported.
  \item \code{ScopeTraits::ClosureType}: valid only for class types.  This trait indicates that the scope represents a closure type.
  \item \code{ScopeTraits::Final}: valid only for class types. This trait indicates that the scope is defined \code{final}.
  \item \code{ScopeTraits::Vendor}: valid only if the scope entity has vendor-defined traits.
\end{itemize}

\subsubsection{Class-type layout packing}
\label{sec:ifc-class-layout-packing}

A value of class layout packing is expressed as value of type \type{PackSize} defined as \newtype{PackSize}{16}

\subsection{\valueTag{DeclSort::Enumeration}}
\label{sec:ifc:DeclSort:Enumeration}

A \type{DeclIndex} abstract reference with tag \valueTag{DeclSort::Enumeration} designates an enumeration declaration.
The \field{index} of that abstract reference is an index into the enumeration declaration partition.
Each entry in that partition is a structure with the following components:
%
\begin{figure}[H]
	\centering
	\structure[text width = 14em]{
		\DeclareMember{name}{TextOffset} \\
		\DeclareMember{locus}{SourceLocation} \\
		\DeclareMember{type}{TypeIndex} \\
		\DeclareMember{base}{TypeIndex} \\
		\DeclareMember{initializer}{Sequence} \\
		\DeclareMember{home\_scope}{DeclIndex} \\
		\DeclareMember{alignment}{ExprIndex} \\
		\DeclareMember{specifiers}{BasicSpecifiers} \\
		\DeclareMember{access}{Access} \\
		\DeclareMember{properties}{ReachableProperties} \\
	}
	\caption{Structure of an enumeration declaration}
	\label{fig:ifc-enumeration-decl-structure}
\end{figure}
%
The \field{name} field designates the name of the enumeration type.
The \field{locus} field designates the source location.
The \field{type} field designates the kind of enumeration, with:
\begin{itemize}
   \item \code{TypeBasis::Enum} meaning a classic enumeration
   \item \code{TypeBasis::Class} or \code{TypeBasis::Struct} meaning a scoped enumeration
\end{itemize}
The \field{base} field designates the underlying type of the enumeration.
The \field{initializer} is a slice (\secref{sec:ifc-sequence}) of the enumerator partition (\secref{sec:ifc:DeclSort:Enumerator}). 
 It designates the sequence of enumerators (if any) declared as part
of the enumeration declaration.
The \field{home\_scope} field designates the declaration of the enclosing scope of the enumeration.
The \field{alignment} designates the alignment specified in the declaration.  A non-zero value indicates an explicit alignment specification
in the input source code.
The \field{specifiers} designates the basic generic declaration specifiers of the enumeration.
The \field{access} designates the access specifier.
The \field{properties} designates the set of reachable semantic properties.

\partition{decl.enum}

\subsection{\valueTag{DeclSort::Alias}} 
\label{sec:ifc:DeclSort:Alias}

A \type{DeclIndex} abstract reference with tag \valueTag{DeclSort::Alias} designates a type alias declaration.
The \field{index} field of that reference is an index into the type alias declaration partition.
Each entry in that partition is a structure with the following components:
%
\begin{figure}[H]
	\centering
	\structure{
		\DeclareMember{name}{TextOffset} \\
		\DeclareMember{locus}{SourceLocation} \\
		\DeclareMember{type}{TypeIndex} \\
		\DeclareMember{home\_scope}{DeclIndex} \\
		\DeclareMember{aliasee}{TypeIndex} \\
		\DeclareMember{specifiers}{BasicSpecifiers} \\
		\DeclareMember{access}{Access} \\
	}
	\caption{Structure of a type alias declaration}
	\label{fig:ifc-type-alias-decl-structure}
\end{figure}
%
\begin{itemize}
	\item The \field{name} field designates the name of the alias.
	\item The \field{locus} field designates the source location.
	\item The \field{type} field denotes the kind of alias: it denotes \code{TypeBasis::Typename} for type aliases; it will be \code{TypeBasis::Namespace} for namespace aliases; it is abstract reference of sort \valueTag{TypeSort::Forall} for template aliases.
	\item The \field{home\_scope} field designates the declaration of the enclosing scope.
	\item The \field{aliasee} field designates the type the alias is declared for.
	\item The \field{specifiers} field designates the basic declaration specifiers for the alias.
	\item The \field{access} field designates the access specifier for the alias.
\end{itemize}

This structure is also used to represent template aliases -- mistakenly called alias templates in the ISO C++ document.  For template aliases,
the \field{aliasee} field denotes a \valueTag{TypeSort::Forall} type (\sortref{Forall}{TypeSort}), which is the representation of the \grammar{template-parameter} list followed by the \grammar{type-id} 
that would syntactically appear on the right hand side of the source-level \grammar{using-declaration}.


\partition{decl.alias}


\subsection{\valueTag{DeclSort::Temploid}}
\label{sec:ifc:DeclSort:Temploid}

A member of a parameterized scope -- does not have template parameters of its own.
%
\begin{figure}[H]
	\centering
	\structure[text width = 14em]{
		\DeclareMember{entity}{ParameterizedEntity}\\
		\DeclareMember{chart}{ChartIndex}\\
		\DeclareMember{properties}{ReachableProperties} \\
	}
	\caption{Structure of a templated declaration}
	\label{fig:ifc-type-temploid-structure}
\end{figure}
%
The \field{entity} field represents the declaration being parameterized.
The \field{chart} field designates the set of template parameter lists of the enclosing scope.
The \field{properties} field designates the set of reachable semantics properties.


\partition{decl.temploid}

\subsubsection{Parameterized Entity}
\label{sec:ifc-parameterized-entity}

The structure \type{ParameterizedEntity} has the following layout
%
\begin{figure}[H]
	\centering
	\structure{
		\DeclareMember{decl}{DeclIndex} \\
		\DeclareMember{head}{SentenceIndex} \\
		\DeclareMember{body}{SentenceIndex} \\
		\DeclareMember{attributes}{SentenceIndex} \\
	}
	\caption{Structure of a declaration parameterized by a template}
	\label{fig:ifc-parameterized-decl-structure}
\end{figure}
%
The \field{decl} field denotes the declaration that is being 
parameterized either directly or indirectly by a set of template parameter lists.
The \field{head} field designates the sentence (\secref{sec:ifc-sentence}) that makes up the
non-defining declarative part of the current instantiation.
That sentence is no longer meaningful in recent releases of MSVC since
any semantics information can be obtained from the entity denoted
by \field{decl}.
The \field{body} field denotes the sentence of the defining ("body") part
of the current instantiation.  This field is meaningful only
for templated functions. 


\subsection{\valueTag{DeclSort::Template}}
\label{sec:ifc:DeclSort:Template}

A template declaration: class, function, constructor, type alias, variable.

\begin{figure}[H]
	\centering
	\structure[text width = 14em]{
		\DeclareMember{name}{NameIndex} \\
		\DeclareMember{locus}{SourceLocation} \\
		\DeclareMember{home\_scope}{DeclIndex} \\
		\DeclareMember{chart}{ChartIndex} \\
		\DeclareMember{entity}{ParameterizedEntity} \\
		\DeclareMember{type}{TypeIndex} \\
		\DeclareMember{specifiers}{BasicSpecifiers} \\
		\DeclareMember{access}{Access} \\
		\DeclareMember{properties}{ReachableProperties} \\
	}
	\caption{Structure of a template declaration}
	\label{fig:ifc-template-decl-structure}
\end{figure}

The \field{name} field denotes the name of this template.
The \field{locus} field denotes the source location of this declaration.
The \field{home\_scope} field designate the home scope of this template.
The \field{chart} field denotes the set of parameter list to this template.
The \field{entity} field describes the declaration being parameterized by this template.  Its structure is defined below.
The \field{type} field denotes the type of this template declaration.
The \field{specifiers} field denotes declaration specifiers for this template.
The \field{access} field denotes the access level of this declaration.
The \field{properties} field designates the set of reachable semantic properties.

\partition{decl.template}

\subsection{\valueTag{DeclSort::PartialSpecialization}}
\label{sec:ifc:DeclSort:PartialSpecialization}

A partial specialization of a
template (class-type or function). 

%
\begin{figure}[H]
	\centering
	\structure[text width = 14em]{
		\DeclareMember{name}{NameIndex} \\
		\DeclareMember{locus}{SourceLocation} \\
		\DeclareMember{home\_scope}{DeclIndex} \\
		\DeclareMember{chart}{ChartIndex} \\
		\DeclareMember{entity}{ParameterizedEntity} \\
		\DeclareMember{form}{Index} \\
		\DeclareMember{specifiers}{BasicSpecifiers} \\
		\DeclareMember{access}{Access} \\
		\DeclareMember{properties}{ReachableProperties} \\
	}
	\caption{Structure of a partial specialization declaration}
	\label{fig:ifc-partial-specialization-decl-structure}
\end{figure}

The \field{name} field denotes the name of the current instantiation of the partial specialization.
The \field{locus} field denotes the source location.
The \field{home\_scope} field denotes the parent declaration of this partial specialization.
The \field{chart} field denotes the set of template-parameter lists of this partial specialization.
The \field{entity} field describes the current instantiation of this partial specialization.
The \field{form} field is an index into the partition of specialization form (template and template-argument list) named \code{"form.spec"}.
The \field{specifiers} field denotes the declaration specifiers of this partial specialization.
The \field{access} field denotes the access level of this partial specialization.
The \field{properties} field denotes the set of reachable semantic properties.

\partition{decl.partial-specialization}


\subsection{\valueTag{DeclSort::ExplicitSpecialization}} 
\label{sec:ifc:DeclSort:ExplicitSpecialization}

An explicit specialization
of a template (class-type or function). 

\begin{figure}[H]
	\centering
	\structure[text width = 14em]{
		\DeclareMember{form}{Index} \\
		\DeclareMember{decl}{DeclIndex} \\
	}
	\caption{Structure of an explicit specialization declaration}
	\label{fig:ifc-explicit-specialization-declaration}
\end{figure}

The \field{form} field is an index into the specialization form partition (named "form.spec").
The \field{decl} field denotes the declaration under the empty template parameter list.
The \field{properties} field designates the set of reachable semantic properties.

\partition{decl.explicit-specialization}


\subsection{\valueTag{DeclSort::ExplicitInstantiation}} 
\label{sec:ifc:DeclSort:ExplicitInstantiation}

An explicit instantiation
request of a template specialization. 
%
\begin{figure}[H]
	\centering
	\structure[text width = 14em]{
		\DeclareMember{form}{Index} \\
		\DeclareMember{decl}{DeclIndex} \\
	}
	\caption{Structure of an explicit instantiation declaration}
	\label{fig:ifc-explicit-instantiation-declaration}
\end{figure}
%

The \field{form} field is an index into the partition of specialization forms.
The \field{decl} field denotes the declarations designated by the explicit instantiation.

\partition{decl.explicit-instantiation}

\subsection{\valueTag{DeclSort::Concept}}
\label{sec:ifc:DeclSort:Concept}

A \type{DeclIndex} value with tag \valueTag{DeclSort::Concept} represents an abstract reference
to a concept declaration.  The \field{index} of that abstract reference is an index into the concept
definition partition.  Each entry in that partition is a structure with the following layout
%
\begin{figure}[H]
	\centering
	\structure{
		\DeclareMember{name}{TextOffset} \\
		\DeclareMember{locus}{SourceLocation} \\
		\DeclareMember{home\_scope}{DeclIndex} \\
		\DeclareMember{type}{TypeIndex} \\
		\DeclareMember{chart}{ChartIndex} \\
		\DeclareMember{constraint}{ExprIndex} \\
		\DeclareMember{specifiers}{BasicSpecifiers} \\
		\DeclareMember{access}{Access} \\
		\DeclareMember{head}{SentenceIndex} \\
		\DeclareMember{body}{SentenceIndex} \\
	}
	\caption{Structure of a concept declaration}
	\label{fig:ifc-concept-decl-structure}
\end{figure}
%
and meaning of the fields:
\begin{itemize}
	\item \field{name} designates the name of the concept
	\item \field{locus} designates the source location of the concept definition
	\item \field{home\_scope} designates the declaration of the enclosing scope
	\item \field{type} designates the full type of the concept definition
	\item \field{chart} is the parameter list list to the concept
	\item \field{constraint} is the body of predicate defining the concept
	\item \field{specifiers} is the set of basic declaration specifiers
	\item \field{access} is the access specifier for the concept definition
	\item \field{head} is the sequence of words making up the declarative part of the concept
	\item \field{body} is the sequence of words making up the body of the concept definition
\end{itemize}

\partition{decl.concept}

\note{This representation is subject to change in future releases.}

\subsection{\valueTag{DeclSort::Function}}
\label{sec:ifc:DeclSort:Function}

A \type{DeclIndex} value with tag \valueTag{DeclSort::Function} represents
an abstract reference to a function declaration. Note that a static member function is represented as a function.
The \field{index} field is an index into the function declaration partition.
Each entry in that partition is a structure with the following components:
%
\begin{figure}[H]
	\centering
	\structure [text width = 15em] {
		\DeclareMember{name}{NameIndex} \\
		\DeclareMember{locus}{SourceLocation} \\
		\DeclareMember{type}{TypeIndex} \\
		\DeclareMember{home\_scope}{DeclIndex} \\
		\DeclareMember{chart}{ChartIndex} \\
		\DeclareMember{traits}{FunctionTraits} \\
		\DeclareMember{specifiers}{BasicSpecifiers} \\
		\DeclareMember{access}{Access} \\
		\DeclareMember{properties}{ReachableProperties} \\
	}
	\caption{Structure of a function declaration}
	\label{fig:ifc-function-decl-structure}
\end{figure}
%
The \field{name} field designates the name of the function.
The \field{locus} field designates the source location.
The \field{type} field designates the type of the function, including the noexcept-specification (which is now part of the type of a function in C++17).
The \field{home\_scope} denotes the enclosing scope of the function. This may not be the lexical scope of the declaration.
The \field{chart} denotes the chart (\secref{sec:ifc-charts}) of the function parameter list along with their default arguments.  These parameters may be unnamed.
The \field{specifier} denotes basic declaration specifiers; the \field{traits} field adds additional function traits.
Finally, the \field{access} denotes the access specifier for the function.

\note{The set of parameter declarations in a function definition is listed in a separate trait (\secref{sec:ifc-msvc-fun-parms}).  That representation is subject to removal in future MSVC releases.}

\partition{decl.function}

\subsubsection{Function traits}
\label{sec:ifc-function-traits}

Certain function-specific cumulative properties are expressed as values of the bitmask type \type{FunctionTraits} defined as
%
\begin{typedef}{FunctionTraits}{}
	enum class FunctionTraits : uint16_t {
		None			= 0,
		Inline			= 1 << 0,
		Constexpr		= 1 << 1,
		Explicit		= 1 << 2,
		Virtual			= 1 << 3,
		NoReturn		= 1 << 4,
		PureVirtual	= 1 << 5,
		HiddenFriend	= 1 << 6,
		Defaulted		= 1 << 7,
		Deleted		= 1 << 8,
		Constrained   = 1 << 9,
		Immediate = 1 << 10,
		Vendor			= 1 << 15,
	};
\end{typedef}
%
with the following meaning
\begin{description}
	\item \code{FunctionTraits::None}: no property
	\item \code{FunctionTraits::Inline}: the function is declared \code{inline}
%	\item \code{FunctionTraits::ForceInline}: the function is declared \code{\_\_forceinline}
	\item \code{FunctionTraits::Constexpr}: the function is declared \code{constexpr}
	\item \code{FunctionTraits::Explicit}: the function is declared \code{explicit}
	\item \code{FunctionTraits::Virtual}: the function is declared \code{virtual}
	\item \code{FunctionTraits::NoReturn}: the function is declared \code{[[noreturn]]} or \code{\_\_declspec(noreturn)}
%	\item \code{FunctionTraits::Naked}: the function is declared \code{\_\_declspec(naked)}
%	\item \code{FunctionTraits::NoAlias}: the function is declared \code{\_\_declspec(noalias)}
%	\item \code{FunctionTraits::NoThrow}: the function is declared \code{\_\_declspec(nothrow)}
%	\item \code{FunctionTraits::NoInline}: the function is declared \code{\_\_declspec(noinline)}
%	\item \code{FunctionTraits::Restrict}: the function is declared \code{\_\_declspec(restrict)}
%	\item \code{FunctionTraits::SafeBuffers}: the function is declared \code{\_\_declspec(safebuffers)}
%	\item \code{FunctionTrsits::DllExport}: the function is declared \code{\_\_declspec(dllexport)}
%	\item \code{FunctionTraits::CodeSegment}: the function is declared \code{\_\_declspec(code\_seg)}
	\item \code{FunctionTraits::PureVirtual}: the function is pure virtual, e.g. with \code{ = 0}
	\item \code{FunctionTraits::HiddenFriend}: the function is a hidden friend
     \item \code{FunctionTraits::Constrained} : the function has requires-constraints
     \item \code{FunctionTraits::Immediate}: the function is a \code{consteval}, or an immediate function. 
     \item \code{FunctionTraits::Vendor}: the function has vendor-defined traits stored in the
	 MSVC vendor-specific traits (\secref{sec:ifc-msvc-vendor-specific-trait}).
\end{description}


\subsection{\valueTag{DeclSort::Method}} 
\label{sec:ifc:DeclSort:Method}

A \type{DeclIndex} abstract reference with tag \code{DeclSort::Method} designates a non-static member function (which is neither a constructor nor a destructor) declaration.
The \field{index} of that abstract reference is an index into the non-static member function declaration partition.
Each entry in that partition is a structure with the following components
%
\begin{figure}[H]
	\centering
	\structure[text width = 15em]{
		\DeclareMember{name}{NameIndex} \\
		\DeclareMember{locus}{SourceLocation} \\
		\DeclareMember{type}{TypeIndex} \\
		\DeclareMember{home\_scope}{DeclIndex} \\
		\DeclareMember{chart}{ChartIndex} \\
		\DeclareMember{traits}{FunctionTraits} \\
		\DeclareMember{specifiers}{BasicSpecifiers} \\
		\DeclareMember{access}{Access} \\
		\DeclareMember{properties}{ReachableProperties} \\
	}
	\caption{Structure of a non-static member function declaration}
	\label{fig:ifc-method-decl-structure}
\end{figure}
%
The \field{name} field designates the name of the non-static member function.
The \field{locus} field designates the source location of this declaration.
The \field{type} field designates the type of this non-static member function.  Note that the type also includes the calling convention.
The \field{home\_scope} designates the enclosing type declaration.
The \field{chart} designates the function parameter list along with their default arguments.
The \field{traits} indicates any additional function-specific traits (\secref{sec:ifc-function-traits}).
The \field{access} designates the access specifier for this function.

\partition{decl.method}


\subsection{\valueTag{DeclSort::Constructor}} 
\label{sec:ifc:DeclSort:Constructor}

A \type{DeclIndex} abstract reference with tag \valueTag{DeclSort::Constructor} designates a constructor declaration.
The \field{index} field of that abstract reference is an index into the constructor declaration partition.
Each entry in that partition is a structure with the following components.
%
\begin{figure}[H]
	\centering
	\structure[text width = 15em]{
		\DeclareMember{name}{TextOffset} \\
		\DeclareMember{locus}{SourceLocation} \\
		\DeclareMember{type}{TypeIndex} \\
		\DeclareMember{home\_scope}{DeclIndex} \\
		\DeclareMember{chart}{ChartIndex} \\
		\DeclareMember{traits}{FunctionTraits} \\
		\DeclareMember{specifiers}{BasicSpecifiers} \\
		\DeclareMember{access}{Access} \\
		\DeclareMember{properties}{ReachableProperties} \\
	}
	\caption{Structure of a constructor declaration}
	\label{fig:ifc-constructor-decl-structure}
\end{figure}
%
The meaning of the fields is as follows:
\begin{itemize}
	\item \field{name} designates the name of the constructor.
	\item \field{locus} denotes the source location of the constructor.
	\item \field{type} denotes the type of this constructor, a type described in \sortref{Tor}{TypeSort}.
	\item \field{home\_scope} denotes the declaration of the enclosing type.
	\item \field{chart} denotes the function parameter list along with their default argument expressions.
	\item \field{traits} designates function-specific traits (\secref{sec:ifc-function-traits}).
	\item \field{specifiers} designate the usual basic declaration specifiers.
	\item \field{access} designates the access specifier of the constructor.
	\item \field{properties} denotes the set of reachable semantic properties of this constructor declaration.
\end{itemize}

\partition{decl.constructor}

\note{The \field{name} field is subject of further design modification}

The structure \type{NoexceptSpecification} has the following layout
%
\begin{figure}[H]
	\centering
	\structure[text width = 15em]{
		\DeclareMember{words}{SentenceIndex} \\
		\DeclareMember{sort}{NoexceptSort} \\
	}
	\caption{Structure of a \code{noexcept}-specification}
	\label{fig:ifc-noexcept-specification-structure}
\end{figure}
%
The \field{words} field denotes the sentence making up the syntax of the noexcept-specification.
This field is meaningful only for templated functions for which the noexcept-specification is a dependent expression.
The \field{sort} field describes the computed semantics, if not dependent.
It has type
\begin{typedef}{NoexceptSort}{}
	enum class NoexceptSort : uint8_t {
		None,
		False,
		True,
		Expression,
		Inferred,
		Unenforced,
	};
\end{typedef}
with the following meaning
\begin{itemize}
	\item \valueTag{NoexceptSort::None}:  No specification is lexically present in the input source
	\item \valueTag{NoexceptSort::False}: The syntax \code{noexcept(false)} was explicitly used, or the determination has similar semantic effect
	\item \valueTag{NoexceptSort::True}: The syntax \code{noexcept(true)} was explicitly used, or the determination has similar semantic effect.
	\item \valueTag{NoexceptSort::Expression}: The syntax \code{noexcept(expr)} was explicitly used, and no determination could be made because the expression is dependent.
	\item \valueTag{NoexceptSort::Inferred}: The \code{noexcept} specification (for a special member) is inferred and dependent on that of the associated functions
		the special member invokes from base class subobjects or non-static data members. 
	\item \valueTag{NoexceptSort::Unenforced}: This is the specification for the static type system, but with no runtime termination enforcement
\end{itemize}

\subsection{\valueTag{DeclSort::InheritedConstructor}}
\label{sec:ifc:DeclSort:InheritedConstructor}

A \type{DeclIndex} abstract reference with tag \valueTag{DeclSort::InheritedConstructor}
designates an inherited constructor.  The \field{index} of that abstract reference
is an index into the inherited constructor partition.  Each entry in that partition
is a structure with the following layout
%
\begin{figure}[H]
	\centering
	\structure[text width = 15em]{
		\DeclareMember{name}{TextOffset} \\
		\DeclareMember{locus}{SourceLocation} \\
		\DeclareMember{type}{TypeIndex} \\
		\DeclareMember{home\_scope}{DeclIndex} \\
		\DeclareMember{chart}{ChartIndex} \\
		\DeclareMember{traits}{FunctionTraits} \\
		\DeclareMember{specifiers}{BasicSpecifiers} \\
		\DeclareMember{access}{Access} \\
		\DeclareMember{base\_ctor}{DeclIndex} \\
	}
\end{figure}
%
The meaning of the fields is as follows
\begin{itemize}
	\item \field{name} designates the name pf this constructor
	\item \field{locus} denotes the source location of this declaration.
	\item \field{type} denotes the type of this constructor, a type described in \sortref{Tor}{TypeSort}.
	\item \field{home\_scope} denotes the declaration of the enclosing type.
	\item \field{chart} denotes the function parameter list along with their default argument expressions.
	\item \field{traits} designates function-specific traits (\secref{sec:ifc-function-traits}).
	\item \field{specifiers} designate the usual basic declaration specifiers.
	\item \field{access} designates the access specifier of the constructor.
	\item \field{base\_ctor} denotes the constructor from the base class this constructor inherits.
\end{itemize}

\partition{decl.inherited-constructor}

\subsection{\valueTag{DeclSort::Destructor}} 
\label{sec:ifc:DeclSort:Destructor}

 A \type{DeclIndex} abstract reference with tag \valueTag{DeclSort::Destructor} designates a destructor declaration.
The \field{index} field of that abstract reference is an index into the destructor declaration partition.
Each entry in that partition is a structure with the following components.
%
\begin{figure}[H]
	\centering
	\structure[text width = 15em]{
		\DeclareMember{name}{TextOffset} \\
		\DeclareMember{locus}{SourceLocation} \\
		\DeclareMember{home\_scope}{DeclIndex} \\
		\DeclareMember{eh\_spec}{NoexceptSpecification} \\ 
		\DeclareMember{traits}{FunctionTraits} \\
		\DeclareMember{specifiers}{BasicSpecifiers} \\
		\DeclareMember{access}{Access} \\
		\DeclareMember{convention}{CallingConvention} \\
		\DeclareMember{properties}{ReachableProperties} \\
	}
	\caption{Structure of a destructor declaration}
	\label{fig:ifc-destructor-decl-structure}
\end{figure}
%
The \field{name} field designates the name of the destructor declaration.
The \field{locus} field designates the source location of the declaration.
The \field{home\_scope} field designates the declaration of the enclosing type.
The \field{eh\_spec} field designates the exception specification of the declaration.
The \field{traits} field designates function-specific properties (\secref{sec:ifc-function-traits}) of the declaration.
The \field{specifiers} field designate the usual basic declaration specifiers.
The \field{access} field designates the access specifier of the destructor declaration.
The \field{convention} field designates the calling convention used by the destructor.


\note{The \field{name} field is subject of further design modification}


\partition{decl.destructor}


\subsection{\valueTag{DeclSort::Reference}} 
\label{sec:ifc:DeclSort:Reference}

A \type{DeclIndex} abstract reference with tag \valueTag{DeclSort::Reference} designates a reference to a declaration
made available by an imported module.
The \field{index} field of that abstract reference is an index into the declaration reference partition.
Each entry in that partition is a structure with the following components:
%
\begin{figure}[H]
	\centering
	\structure{
		\DeclareMember{unit}{ModuleReference} \\
		\DeclareMember{local\_index}{DeclIndex} \\
	}
	\caption{Structure of a declaration reference declaration}
	\label{fig:ifc-reference-decl-structure}
\end{figure}
%
The \field{unit} field designates the owning translation unit.
The \field{local\_index} is the \type{DeclIndex} abstract reference assigned to that entity by the current (importing) module unit.

\partition{decl.reference}

\subsubsection{\type{ModuleReference}}

The type \type{ModuleReference} is a structure with the following layout
\begin{figure}[H]
	\centering
	\structure{
		\DeclareMember{owner}{TextOffset} \\
		\DeclareMember{partition}{TextOffset} \\
	}
	\caption{Structure of a module reference}
	\label{fig:ifc-module-reference-decl-structure}
\end{figure}

Then the \field{owner} field is null, then it means the IFC comes from the
global module, and the \field{partition} field is the name of the source file
out of which the header unit was built.  Otherwise, \field{owner} is the name
of the owning module of the IFC, and the \field{partition} designates the module
partition when it is not null.

\subsection{\valueTag{DeclSort::UsingDeclaration}} 
\label{sec:ifc:DeclSort:UsingDeclaration}

A \type{DeclIndex} abstract reference with tag \valueTag{DeclSort::UsingDeclaration} designates a using declaration.
The \field{index} field of that abstract reference is an index into the using declaration partition.
Each entry in that partition is a structure with the following components:
%
\begin{figure}[H]
	\centering
	\structure{
		\DeclareMember{name}{NameIndex} \\
		\DeclareMember{locus}{SourceLocation} \\
		\DeclareMember{home\_scope}{DeclIndex} \\
		\DeclareMember{resolution}{DeclIndex} \\
		\DeclareMember{parent}{ExprIndex} \\
		\DeclareMember{name2}{TextOffset} \\
		\DeclareMember{specifiers}{BasicSpecifiers} \\
		\DeclareMember{access}{Access} \\
		\DeclareMember{hidden}{bool} \\
	}
	\caption{Structure of a using-declaration structure}
	\label{fig:ifc-using-declaration-structure}
\end{figure}
%

The \field{name} field is the unqualified part of the using declaration.
The \field{locus} field is the source location of the declaration.
The \field{home\_scope} field designates the enclosing scope of the declaration.
The \field{resolution} field designates the set of used declarations: either a single declaration, or a tuple of declarations.
The \field{parent} field designates the qualifying part of the declaration.
The \field{name2} is a redundant field referring to the name of the member.
The \field{specifiers} field denotes the basic declaration specifiers.
The \field{hidden} field indicates whether the member is hidden.

\partition{decl.using-declaration}

\note{This representation is subject to change.}


\subsection{\valueTag{DeclSort::UsingDirective}}
\label{sec:ifc:DeclSort:UsingDirective}

A \type{DeclIndex} abstract reference with tag \valueTag{DeclSort::UsingDirective} designates a using directive.


%
\begin{figure}[H]
	\centering
	TBD.
\end{figure}
%

\note{The structure of this declaration is not yet defined.}

\partition{decl.using-directive}


\subsection{\valueTag{DeclSort::Friend}}
\label{sec:ifc:DeclSort:Friend}

A \type{DeclIndex} abstract reference with tag \valueTag{DeclSort::Friend} designates a friend declaration.
The \field{index} field of that abstract reference is an index into the friend declaration partition.
Each entry in that partition is a structure with the following component:
%
\begin{figure}[H]
	\centering
	\structure{
		\DeclareMember{entity}{ExprIndex}\\
	}
	\caption{Structure of a friend declaration}
	\label{fig:ifc:DeclSort:Friend}
\end{figure}
%

The \field{entity} field denotes and expression that references the declared friend.  For instance, \field{entity}
would denote a named declaration (\sortref{NamedDecl}{ExprSort}) if the input source-level friend declaration references
a lexically declared entity.  If that friend is a specialization of a template then \field{entity} would denote 
the corresponding \grammar{template-id} (\sortref{TemplateId}{ExprSort}).

\partition{decl.friend}


\subsection{\valueTag{DeclSort::Expansion}}
\label{sec:ifc:DeclSort:Expansion}

A \type{DeclIndex} abstract reference with tag \valueTag{DeclSort::Expansion} designates
a declaration obtained by expanding a pack.  The \field{index} field is an index
into the expansion declaration partition.  Each entry in that partition is a structure
with the following layout
%
\begin{figure}[H]
	\centering
	\structure{
		\DeclareMember{operand}{DeclIndex} \\
		\DeclareMember{locus}{SourceLocation} \\
	}
	\caption{Structure of an expansion declaration}
	\label{fig:ifc-expansion-declaration-structure}
\end{figure}
%

\partition{decl.expansion}


\subsection{\valueTag{DeclSort::DeductionGuide}}
\label{sec:ifc:DeclSort:DeductionGuide}

A \type{DeclIndex} abstract reference with tag \valueTag{DeclSort::DeductionGuide} designates
a deduction guide declaration.  The \field{index} field is an index into the 
deduction guide partition.  Each entry in that partition is a structure with the 
following layout
%
\begin{figure}[H]
	\centering
	\structure{
		\DeclareMember{name}{NameIndex} \\
		\DeclareMember{locus}{SourceLocation} \\
		\DeclareMember{parameters}{ChartIndex} \\
		\DeclareMember{specialization}{TypeIndex} \\
		\DeclareMember{traits}{FunctionTraits} \\
		\DeclareMember{specifiers}{BasicSpecifiers} \\
		\DeclareMember{access}{Access} \\
	}
	\caption{Structure of a deduction guide declaration}
	\label{fig:ifc-deduction-guide-decl-structure}
\end{figure}

\partition{decl.deduction-guide}


\subsection{\valueTag{DeclSort::Barren}}
\label{sec:ifc:DeclSort:Barren}

A \type{DeclIndex} abstract reference with tag \valueTag{DeclSort::Barren} designates
a declaration that introduces no name, e.g. \grammar{asm-declaration}, \grammar{static\_assert-declaration}, \grammar{attribute-declaration}, \grammar{empty-declaration}.
The \field{index} field is an index into the barren declaration partition.  Each entry
in that partition is a structure with the following layout
%
\begin{figure}[H]
	\centering
	TBD.
\end{figure}
%

\partition{decl.barren}


\subsection{\valueTag{DeclSort::Tuple}}
\label{sec:ifc:DeclSort:Tuple}

A \type{DeclIndex} abstract reference with tag \valueTag{DeclSort::Tuple} designates a sequence
of declarations referenced in other source-level constructs, e.g. a set of bindings during 
parsing of templated declarations.  The \field{index} field is an index into the
tuple declaration partition.  Each entry in that partition is a structure with the following
layout
%
\begin{figure}[H]
	\centering
	\structure{
		\DeclareMember{start}{Index} \\
		\DeclareMember{cardinality}{Cardinality} \\
	}
	\caption{Structure of a tuple declaration}
	\label{fig:ifc-tuple-declaration-structure}
\end{figure}
%
The \field{start} field is an index into the declaration heap 
partition (\secref{sec:ifc-decl-heap}) pointing to the first declaration in 
the tuple sequence.  The \field{cardinality} field
denotes the number of declaration in the tuple sequence.

\partition{decl.tuple}



\subsection{\valueTag{DeclSort::SyntaxTree}}
\label{sec:ifc:DeclSort:SyntaxTree}

A syntax tree in template declaration.
\begin{figure}[H]
	\centering
	TBD
\end{figure}

\partition{decl.syntax-tree}

\note{This is a vendor extension, the meta description of which is subject to change}


\subsection{\valueTag{DeclSort::Intrinsic}} 
\label{sec:ifc:DeclSort:Intrinsic}

A \type{DeclIndex} abstract reference with tag \valueTag{DeclSort::Intrinsic} designates an intrinsic function declaration.
The \field{index} field of that abstract reference is an index into the intrinsic function declaration partition.
Each entry in that partition is a structure with the following components:
%
\begin{figure}[H]
	\centering
	\structure{
		\DeclareMember{name}{TextOffset} \\
		\DeclareMember{locus}{SourceLocation} \\
		\DeclareMember{type}{TypeIndex} \\
		\DeclareMember{home\_scope}{DeclIndex} \\
		\DeclareMember{specifiers}{BasicSpecifiers} \\
		\DeclareMember{access}{Access} \\
	}
	\caption{Structure of an intrinsic function declaration}
	\label{fig:ifc-intrinsic-decl-structure}
\end{figure}
%

\partition{decl.intrinsic}

\note{This is a vendor extension, the meta description of which is subject to change}


\subsection{\valueTag{DeclSort::Property}}
\label{sec:ifc:DeclSort:Property}

A \type{DeclIndex} abstract reference with tag \valueTag{DeclSort::Property}
designates MSVC's extension of ``property declaration.''  The \field{index}
field is an index into the property declaration partition.  Each entry in that partition
is a structure with the following layout
%
\begin{figure}[H]
	\centering
	\structure{
		\DeclareMember{member}{DeclIndex} \\
		\DeclareMember{getter}{TextOffset} \\
		\DeclareMember{setter}{TextOffset} \\
	}
	\caption{Structure of a property declaration}
	\label{fig:ifc-property-declaration-structure}
\end{figure}
%
The \field{member} field designates the declaration of the non-static data member
being declared a property.  The \field{getter} and the \field{setter} fields 
denote the names of the getters and the setters non-static member functions associated
with the property.

\partition{decl.property}

\note{This is a vendor extension, the meta description of which is subject to change}


\subsection{\valueTag{DeclSort::OutputSegment}}
\label{sec:ifc:DeclSort:OutputSegment}

 Code segment. These are 'declared' via pragmas.

 A \type{DeclIndex} abstract reference with tag \valueTag{DeclSort::OutputSegment}
 designates an MSVC extension declaration of a code segment --- typically declared with a pragma.
 The \field{index} field is an index into the code segment partition.  Each entry
 in that partition is a structure with the following layout
 %
 \begin{figure}[H]
	\centering
	\structure{
		\DeclareMember{name}{TextOffset} \\
		\DeclareMember{ID}{TextOffset} \\
		\DeclareMember{traits}{SegmentTraits} \\
		\DeclareMember{type}{SegmentType} \\
	}
	\caption{Structure of a code segment declaration}
	\label{fig:ifc-code-segment-structure}
 \end{figure}
 %

 The \field{name} field denotes the name of the code segment.
 The \field{ID} field denotes the class-ID of the code segment.
 The \field{traits} field denotes the set of traits of the code segment.
 The \field{type} field denotes the code segment type.

\subsubsection{Code segment traits}
\label{sec:ifc-code-segment-traits}

Each code segment is associated with a set of traits encoded as a value
of type
%
\begin{typedef}{SegmentTraits}{}
	enum class SegmentTraits : uint32_t { };
\end{typedef}
%

\subsubsection{Code segment type}
\label{sec:ifc-code-segment-type}

Each code segment is associated with a ``code segment type'', a value of type
%
\begin{typedef}{SegmentType}{}
	enum class SegmentType : uint8_t { };
\end{typedef}

\partition{decl.segment}

\note{This is a vendor extension, the meta description of which is subject to change}
