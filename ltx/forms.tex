\label{sec:ifc-preprocessing}


The header unit specification requires a C++ translator to save the preprocessing
 state as active at the end of processing a header source file, and reifying that 
 state upon import declaration.  The MSVC toolset now requires the selection 
 of conformant preprocessor mode (\code{/Zc:preprocessor}) when processing header units, 
 either at the header unit construction time or at import time.

\section{Macro definitions}
\label{sec:ifc-macro-defs}

C++ preprocessor macro definitions are indicated by macro abstract references.
This document uses \type{MacroIndex} as a typed abstract reference
to designate a macro definition.  Like all abstract references, it is a $32$-bit value
\begin{figure}[htbp]
    \centering
    \absref{1}{MacroSort}
    \caption{\type{MacroIndex}: Abstract reference of macro definition}
    \label{fig:ifc-macro-index}
\end{figure}

\begin{SortEnum}{MacroSort}
  \enumerator{ObjectLike}
  \enumerator{FunctionLike}
\end{SortEnum}

\subsection{\valueTag{MacroSort::ObjectLike}}
\label{sec:ifc:MacroSort:ObjectLike}

\begin{figure}[H]
    \centering
    \structure{
        \DeclareMember{locus}{SourceLocation} \\
        \DeclareMember{name}{TextOffset} \\
        \DeclareMember{body}{FormIndex} \\
    }
    \caption{Structure of an object-like macro definition}
    \label{fig:ifc-object-like-macro-structure}
\end{figure}

The field \field{name} designates the name of the macro, and the field \field{body} designates its replacement list.

\partition{macro.object-like}


\subsection{\valueTag{MacroSort::FunctionLike}}
\label{sec:ifc:MacroSort:FunctionLike}

\begin{figure}[H]
    \centering
    \structure{
        \DeclareMember{locus}{SourceLocation} \\
        \DeclareMember{name}{TextOffset} \\
        \DeclareMember{parameters}{FormIndex} \\
        \DeclareMember{body}{FormIndex} \\
        \DeclareMember{arity}{u31} \\
        \DeclareMember{variadic}{u1} \\
    }
    \caption{Structure of a function-like macro definition}
    \label{fig:ifc-function-like-macro-structure}
\end{figure}

The field \field{name} designates the name of the macro.  The field \field{parameters}, when not null,
designate the parameter parameter list.  The field \field{boy} designates the replacement list of the macro.
The field \field{arity} designates the number of named parameter in the parameter list --- this field is in
some sense redundant with the \field{parameters} field.  The field \field{variadic} indicates whether
this macro is variadic.

\partition{macro.function-like}


\section{Preprocessing forms}
\label{sec:ifc-pp-forms}

During the processing of input source files from translation phases 1 through 4, 
syntactic elements (\grammar{preprocessing-token}s) 
are grouped into \emph{preprocessing forms},
denoted by abstract references of type \type{FormIndex} with the following layout:
%
\begin{figure}[htbp]
  \centering
	\absref{4}{FormSort}
  \caption{\type{FormIndex}: Abstract reference of preprocessing forms}
  \label{fig:ifc-form-index}
\end{figure}
%

\begin{SortEnum}{FormSort}
  \enumerator{Identifier}
  \enumerator{Number}
  \enumerator{Character}
  \enumerator{String}
  \enumerator{Operator}
  \enumerator{Keyword}
  \enumerator{Whitespace}
  \enumerator{Parameter}
  \enumerator{Stringize}
  \enumerator{Catenate}
  \enumerator{Pragma}
  \enumerator{Header}
  \enumerator{Parenthesized}
  \enumerator{Tuple}
  \enumerator{Junk}
\end{SortEnum}

\subsection{Form structures}
\label{sec:ifc-form-structures}

\subsubsection{\valueTag{FormSort::Identifier}}
\label{sec:ifc:FormSort:Identifier}

A \type{FormIndex} value with tag \valueTag{FormSort::Identifier} designates
the representation of a \grammar{preprocessing-token} that 
is an \grammar{identifier}.  The \field{index} field is an index into
the preprocessing identifier form partition.  Each structure in that partition
has the following layout
%
\begin{figure}[H]
  \centering
  \structure{
    \DeclareMember{locus}{SourceLocation} \\
    \DeclareMember{spelling}{TextOffset} \\
  }
  \caption{Structure of an identifier form}
  \label{fig:ifc-identifier-form}
\end{figure}

The field \field{locus} designates the source location of this 
\grammar{identifier}.  The
field \field{spelling} designates the sequence of characters making up that 
\grammar{identifier}. 

\partition{pp.ident}

\subsubsection{\valueTag{FormSort::Number}}
\label{sec:ifc:FormSort:Number}

A \type{FormIndex} value with tag \valueTag{FormSort::Number} designates
the representation of a \grammar{preprocessing-token} that 
is a \grammar{pp-number}.  The \field{index} field is an index into
the preprocessing number form partition.  Each structure in that partition
has the following layout
%
\begin{figure}[H]
  \centering
  \structure{
    \DeclareMember{locus}{SourceLocation} \\
    \DeclareMember{spelling}{TextOffset} \\
  }
  \caption{Structure of a number form}
  \label{fig:ifc-number-form}
\end{figure}

The field \field{locus} designates the source location of this form.  The
field \field{spelling} designates the sequence of characters making up that 
\grammar{pp-number}. 


\partition{pp.num}

\subsubsection{\valueTag{FormSort::Character}}
\label{sec:ifc:FormSort:Character}

A \type{FormIndex} value with tag \valueTag{FormSort::Character} designates
the representation of a \grammar{preprocessing-token} that 
is a \grammar{character-literal} or \grammar{user-defined-character-literal}.
The \field{index} field is an index into
the preprocessing character form partition.  Each structure in that partition
has the following layout
%
\begin{figure}[H]
  \centering
  \structure{
    \DeclareMember{locus}{SourceLocation} \\
    \DeclareMember{spelling}{TextOffset} \\
  }
  \caption{Structure of a character form}
  \label{fig:ifc-character-form}
\end{figure}

The field \field{locus} designates the source location of this form.  The
field \field{spelling} designates the sequence of characters making up that 
\grammar{character-literal} or \grammar{user-defined-character-literal}. 

\partition{pp.char}

\subsubsection{\valueTag{FormSort::String}}
\label{sec:ifc:FormSort:String}

A \type{FormIndex} value with tag \valueTag{FormSort::String} designates
the representation of a \grammar{preprocessing-token} that 
is a \grammar{string-literal} or \grammar{user-defined-string-literal}.
The \field{index} field is an index into
the preprocessing string form partition.  Each structure in that partition
has the following layout
%
\begin{figure}[H]
  \centering
  \structure{
    \DeclareMember{locus}{SourceLocation} \\
    \DeclareMember{spelling}{TextOffset} \\
  }
  \caption{Structure of a string form}
  \label{fig:ifc-string-form}
\end{figure}

The field \field{locus} designates the source location of this form.  The
field \field{spelling} designates the sequence of characters making up that 
\grammar{string-literal} or \grammar{user-defined-string-literal}. 

\partition{pp.string}

\subsubsection{\valueTag{FormSort::Operator}}
\label{sec:ifc:FormSort:Operator}

A \type{FormIndex} value with tag \valueTag{FormSort::Operator} designates
the representation of a \grammar{preprocessing-token} that 
is a \grammar{preprocessing-op-or-punc}.
The \field{index} field is an index into
the preprocessing operator form partition.  Each structure in that partition
has the following layout
%
\begin{figure}[H]
  \centering
  \structure{
    \DeclareMember{locus}{SourceLocation} \\
    \DeclareMember{spelling}{TextOffset} \\
    \DeclareMember{operator}{FormOperator} \\
  }
  \caption{Structure of an operator form}
  \label{fig:ifc-operator-form}
\end{figure}

The field \field{locus} designates the source location of this form.  The
field \field{spelling} designates the sequence of characters making up that 
\grammar{preprocessing-op-or-punc}.  The field \field{operator} designates the
soure-level preprocessing interpretation of that form. 

\partition{pp.op}

\subsubsection{\valueTag{FormSort::Keyword}}
\label{sec:ifc:FormSort:Keyword}

A \type{FormIndex} value with tag \valueTag{FormSort::Keyword} designates
the representation of a \grammar{preprocessing-token} that 
is \grammar{import-keyword}, or \grammar{module-keyword}, 
or \grammar{export-keyword}.
The \field{index} field is an index into
the preprocessing keyword form partition.  Each structure in that partition
has the following layout
%
\begin{figure}[H]
  \centering
  \structure{
    \DeclareMember{locus}{SourceLocation} \\
    \DeclareMember{spelling}{TextOffset} \\
  }
  \caption{Structure of a keyword form}
  \label{fig:ifc-keyword-form}
\end{figure}

The field \field{locus} designates the source location of this form.  The
field \field{spelling} designates the sequence of characters making up that 
keyword: \grammar{import-keyword}, or \grammar{module-keyword}, or
\grammar{export-keyword}.  Note that other standard C++ source-level keywords do
not existing during the preprocessing phases.

\partition{pp.key}

\subsubsection{\valueTag{FormSort::Whitespace}}
\label{sec:ifc:FormSort:Whitespace}

\begin{figure}[H]
  \centering
  \structure{
    \DeclareMember{locus}{SourceLocation} \\
  }
  \caption{Structure of a whitespace form}
  \label{fig:ifc-whitespace-form}
\end{figure}

The field \field{locus} designates the source location of this form.  The
field \field{spelling} designates the sequence of characters making up that 
whitespace.

\note{The current version of the MSVC toolset does not emit whitespace forms}

\partition{pp.space}

\subsubsection{\valueTag{FormSort::Parameter}}
\label{sec:ifc:FormSort:Parameter}

A \type{FormIndex} value with tag \valueTag{FormSort::Parameter} designates
the representation of a macro paramater.
The \field{index} field is an index into
the preprocessing parameter form partition.  Each structure in that partition
has the following layout
%
\begin{figure}[H]
  \centering
  \structure{
    \DeclareMember{locus}{SourceLocation} \\
    \DeclareMember{spelling}{TextOffset} \\
  }
  \caption{Structure of a parameter form}
  \label{fig:ifc-parameter-form}
\end{figure}

The field \field{locus} designates the source location of this form.  The
field \field{spelling} designates the sequence of characters making up that 
parameter name.

\partition{pp.param}

\subsubsection{\valueTag{FormSort::Stringize}}
\label{sec:ifc:FormSort:Stringize}

A \type{FormIndex} value with tag \valueTag{FormSort::Stringize} designates
the representation of the application of the stringizing operator (\code{\#}) to
a preprocessing form.
The \field{index} field is an index into
the preprocessing stringizing form partition.  Each structure in that partition
has the following layout
%
\begin{figure}[H]
  \centering
  \structure{
    \DeclareMember{locus}{SourceLocation} \\
    \DeclareMember{operand}{FormIndex} \\
  }
  \caption{Structure of a stringizing form}
  \label{fig:ifc-stringize-form}
\end{figure}

The field \field{locus} designates the source location of this form.  The
field \field{operand} designates the form that is operand to the 
stringizing operator.

\partition{pp.to-string}

\subsubsection{\valueTag{FormSort::Catenate}}
\label{sec:ifc:FormSort:Catenate}

A \type{FormIndex} value with tag \valueTag{FormSort::Catenate} designates
the representation of the application of the token catenation operator
 (\code{\#\#}) to two preprocessing forms.
The \field{index} field is an index into
the preprocessing catenation form partition.  Each structure in that partition
has the following layout
%
\begin{figure}[H]
  \centering
  \structure{
    \DeclareMember{locus}{SourceLocation} \\
    \DeclareMember{first}{FormIndex} \\
    \DeclareMember{second}{FormIndex} \\
  }
  \caption{Structure of a catenation form}
  \label{fig:ifc-catenate-form}
\end{figure}

The field \field{locus} designates the source location of this form.  The
fields \field{first} and \field{second} designate the operands to the 
catenation operator \code{\#\#}.

\partition{pp.catenate}

\subsubsection{\valueTag{FormSort::Pragma}}
\label{sec:ifc:FormSort:Pragma}

A \type{FormIndex} value with tag \valueTag{FormSort::Pragma} designates the
representation of the application of the standard \code{_Pragma} operator or
the MSVC extension \code{__pragma} to a preprocessing form.
The \field{index} field is an index into
the preprocessing pragma form partition.  Each structure in that partition
has the following layout
%
\begin{figure}[H]
  \centering
  \structure{
    \DeclareMember{locus}{SourceLocation} \\
    \DeclareMember{operand}{FormIndex} \\
  }
  \caption{Structure of a pragma form}
  \label{fig:ifc-pragma-form}
\end{figure}

The field \field{locus} designates the source location of this form.  The
field \field{operand} designates the form that is argument to the pragma operator.
If the operand's tag is \valueTag{FormSort::Tuple} then the pragma is actually
the MSVC extension \code{__pragma}, otherwise the tag must be 
\valueTag{FormSort::String} indicating the standard \code{_Pragma} operator.

\partition{pp.pragma}

\subsubsection{\valueTag{FormSort::Header}}
\label{sec:ifc:FormSort:Header}

A \type{FormIndex} value with tag \valueTag{FormSort::Header} designates
the representation of a \grammar{preprocessing-token} that 
is a \grammar{header-name}.
The \field{index} field is an index into
the preprocessing header-name form partition.  Each structure in that partition
has the following layout
%
\begin{figure}[H]
  \centering
  \structure{
    \DeclareMember{locus}{SourceLocation} \\
    \DeclareMember{spelling}{Textoffset} \\
  }
  \caption{Structure of a header form}
  \label{fig:ifc-header-form}
\end{figure}

The field \field{locus} designates the source location of this form.  The
field \field{spelling} designates the spelling of the header-name 
that is argument to an \code{\#include} or \code{import} directive.

\partition{pp.header}

\note{The current MSVC toolset release does not emit this form.}

\subsubsection{\valueTag{FormSort::Parenthesized}}
\label{sec:ifc:FormSort:Parenthesized}

A \type{FormIndex} value with tag \valueTag{FormSort::Parenthesized} designates
the representation of a preprocessing form enclosed in a matching pair 
of parentheses.
The \field{index} field is an index into
the preprocessing parenthesized form partition.  Each structure in that partition
has the following layout
%
\begin{figure}[H]
  \centering
  \structure{
    \DeclareMember{locus}{SourceLocation} \\
    \DeclareMember{operand}{FormIndex} \\
  }
  \caption{Structure of a parenthesized form}
  \label{fig:ifc-parenthesized-form}
\end{figure}

The field \field{locus} designates the source location of this form.  The
field \field{operand} designates the form that is enclosed in the matching 
parenthesis.

\partition{pp.paren}


\subsubsection{\valueTag{FormSort::Tuple}}
\label{sec:ifc:FormSort:Tuple}

A \type{FormIndex} value with tag \valueTag{FormSort::Tuple} designates the
representation of a sequence of preprocessing forms.
The \field{index} field is an index into
the preprocessing tuple form partition.  Each structure in that partition
has the following layout
%
\begin{figure}[H]
  \centering
  \structure{
    \DeclareMember{start}{Index} \\
    \DeclareMember{cardinality}{Cardinality} \\
  }
  \caption{Structure of a tuple form}
  \label{fig:ifc-tuple-form}
\end{figure}

The field \field{locus} designates the source location of this form.  The
field \field{start} designates the position of the \type{FormIndex} value
in the form heap partition that starts the sequence.
The field \field{cardinality} designates the number of \type{FormIndex} values
in the sequence.

\partition{pp.tuple}

\subsubsection{\valueTag{FormSort::Junk}}
\label{sec:ifc:FormSort:Junk}

A \type{FormIndex} value  with tag \valueTag{FormSort::Junk} designates the 
representation of an invalid \grammar{preprocessing-token}.
The \field{index} field is an index into
the preprocessing junk form partition.  Each structure in that partition
has the following layout
%
\begin{figure}[H]
  \centering
  \structure{
    \DeclareMember{locus}{SourceLocation} \\
    \DeclareMember{spelling}{Textoffset} \\
  }
  \caption{Structure of a junk form}
  \label{fig:ifc-junk-form}
\end{figure}

The field \field{locus} designates the source location of this form.  The
field \field{spelling} designates the sequence of characters making up
this non-\grammar{preprocessing-token}

\partition{pp.junk}

\note{The current MSVC toolset does not produce this forms since an IFC is produced
only for a successful translation}

\section{Preprocessing operators}
\label{sec:ifc-preprocessing-operators}

Preprocessing form operators (\grammar{preprocessing-op-or-punc}) are
represented as $16$-bit values with the following layout
%
\begin{figure}[H]
	\centering
	  \begin{BasicAbstractReferenceLayout}{16}
		  \bitfield{3}{\field{sort}}
				\bitfield{13}{\field{value}}
		  \bitFormatTextAt{2}{2}
		   \bitSeparate{3}
		  \bitFormatTextAt{3}{3}
	  \end{BasicAbstractReferenceLayout} 
	  \caption{\type{Operator}: Structure of preprocessing form operator}
	  \label{fig:ifc-preprocessing-operator}
\end{figure}
%
The field \field{sort} is a $3$-bit value of type \clipType{WorSort}{3}
(\secref{sec:ifc:word-structures}), holding either \valueTag{WordSort::Punctuator}
(\secref{sec:ifc:WordSort:Punctuator}) 
or \valueTag{WordSort::Operator} (\secref{sec:ifc:WordSort:Operator}).
The field \field{value} is a $13$-bit value the interpretation of which 
is \field{sort}-dependent: its type is \clipType{SourcePunctuator}{13} 
(\figref{fig:ifc-SourcePunctuator-def})
if \field{sort} is \valueTag{WordSort::Punctuator}; 
otherwise its type is
\clipType{SourceOperator}{13} (\figref{fig:ifc-SourceOperator-def}) 
if \field{sort} is \valueTag{WordSort::Operator}.

\section{Pragmas}
\label{sec:ifc-pragma-form}


Pragmas are indicated by abstract pragma references.  This document
uses \type{PragmaIndex} to designate a typed abstract reference to a
pragma.  Like all abstract references, it is a 32-bit value
\begin{figure}[htbp]
  \centering
  \absref{1}{PragmaSort}
  \caption{\type{PragmaIndex}: Abstract reference of pragmas}
  \label{fig:ifc-pragma-index}
\end{figure}

\begin{SortEnum}{PragmaSort}
	\enumerator{VendorExtension}
\end{SortEnum}
